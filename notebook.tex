
% Default to the notebook output style

    


% Inherit from the specified cell style.




    
\documentclass[11pt]{article}

    
    
    \usepackage[T1]{fontenc}
    % Nicer default font (+ math font) than Computer Modern for most use cases
    \usepackage{mathpazo}

    % Basic figure setup, for now with no caption control since it's done
    % automatically by Pandoc (which extracts ![](path) syntax from Markdown).
    \usepackage{graphicx}
    % We will generate all images so they have a width \maxwidth. This means
    % that they will get their normal width if they fit onto the page, but
    % are scaled down if they would overflow the margins.
    \makeatletter
    \def\maxwidth{\ifdim\Gin@nat@width>\linewidth\linewidth
    \else\Gin@nat@width\fi}
    \makeatother
    \let\Oldincludegraphics\includegraphics
    % Set max figure width to be 80% of text width, for now hardcoded.
    \renewcommand{\includegraphics}[1]{\Oldincludegraphics[width=.8\maxwidth]{#1}}
    % Ensure that by default, figures have no caption (until we provide a
    % proper Figure object with a Caption API and a way to capture that
    % in the conversion process - todo).
    \usepackage{caption}
    \DeclareCaptionLabelFormat{nolabel}{}
    \captionsetup{labelformat=nolabel}

    \usepackage{adjustbox} % Used to constrain images to a maximum size 
    \usepackage{xcolor} % Allow colors to be defined
    \usepackage{enumerate} % Needed for markdown enumerations to work
    \usepackage{geometry} % Used to adjust the document margins
    \usepackage{amsmath} % Equations
    \usepackage{amssymb} % Equations
    \usepackage{textcomp} % defines textquotesingle
    % Hack from http://tex.stackexchange.com/a/47451/13684:
    \AtBeginDocument{%
        \def\PYZsq{\textquotesingle}% Upright quotes in Pygmentized code
    }
    \usepackage{upquote} % Upright quotes for verbatim code
    \usepackage{eurosym} % defines \euro
    \usepackage[mathletters]{ucs} % Extended unicode (utf-8) support
    \usepackage[utf8x]{inputenc} % Allow utf-8 characters in the tex document
    \usepackage{fancyvrb} % verbatim replacement that allows latex
    \usepackage{grffile} % extends the file name processing of package graphics 
                         % to support a larger range 
    % The hyperref package gives us a pdf with properly built
    % internal navigation ('pdf bookmarks' for the table of contents,
    % internal cross-reference links, web links for URLs, etc.)
    \usepackage{hyperref}
    \usepackage{longtable} % longtable support required by pandoc >1.10
    \usepackage{booktabs}  % table support for pandoc > 1.12.2
    \usepackage[inline]{enumitem} % IRkernel/repr support (it uses the enumerate* environment)
    \usepackage[normalem]{ulem} % ulem is needed to support strikethroughs (\sout)
                                % normalem makes italics be italics, not underlines
    

    
    
    % Colors for the hyperref package
    \definecolor{urlcolor}{rgb}{0,.145,.698}
    \definecolor{linkcolor}{rgb}{.71,0.21,0.01}
    \definecolor{citecolor}{rgb}{.12,.54,.11}

    % ANSI colors
    \definecolor{ansi-black}{HTML}{3E424D}
    \definecolor{ansi-black-intense}{HTML}{282C36}
    \definecolor{ansi-red}{HTML}{E75C58}
    \definecolor{ansi-red-intense}{HTML}{B22B31}
    \definecolor{ansi-green}{HTML}{00A250}
    \definecolor{ansi-green-intense}{HTML}{007427}
    \definecolor{ansi-yellow}{HTML}{DDB62B}
    \definecolor{ansi-yellow-intense}{HTML}{B27D12}
    \definecolor{ansi-blue}{HTML}{208FFB}
    \definecolor{ansi-blue-intense}{HTML}{0065CA}
    \definecolor{ansi-magenta}{HTML}{D160C4}
    \definecolor{ansi-magenta-intense}{HTML}{A03196}
    \definecolor{ansi-cyan}{HTML}{60C6C8}
    \definecolor{ansi-cyan-intense}{HTML}{258F8F}
    \definecolor{ansi-white}{HTML}{C5C1B4}
    \definecolor{ansi-white-intense}{HTML}{A1A6B2}

    % commands and environments needed by pandoc snippets
    % extracted from the output of `pandoc -s`
    \providecommand{\tightlist}{%
      \setlength{\itemsep}{0pt}\setlength{\parskip}{0pt}}
    \DefineVerbatimEnvironment{Highlighting}{Verbatim}{commandchars=\\\{\}}
    % Add ',fontsize=\small' for more characters per line
    \newenvironment{Shaded}{}{}
    \newcommand{\KeywordTok}[1]{\textcolor[rgb]{0.00,0.44,0.13}{\textbf{{#1}}}}
    \newcommand{\DataTypeTok}[1]{\textcolor[rgb]{0.56,0.13,0.00}{{#1}}}
    \newcommand{\DecValTok}[1]{\textcolor[rgb]{0.25,0.63,0.44}{{#1}}}
    \newcommand{\BaseNTok}[1]{\textcolor[rgb]{0.25,0.63,0.44}{{#1}}}
    \newcommand{\FloatTok}[1]{\textcolor[rgb]{0.25,0.63,0.44}{{#1}}}
    \newcommand{\CharTok}[1]{\textcolor[rgb]{0.25,0.44,0.63}{{#1}}}
    \newcommand{\StringTok}[1]{\textcolor[rgb]{0.25,0.44,0.63}{{#1}}}
    \newcommand{\CommentTok}[1]{\textcolor[rgb]{0.38,0.63,0.69}{\textit{{#1}}}}
    \newcommand{\OtherTok}[1]{\textcolor[rgb]{0.00,0.44,0.13}{{#1}}}
    \newcommand{\AlertTok}[1]{\textcolor[rgb]{1.00,0.00,0.00}{\textbf{{#1}}}}
    \newcommand{\FunctionTok}[1]{\textcolor[rgb]{0.02,0.16,0.49}{{#1}}}
    \newcommand{\RegionMarkerTok}[1]{{#1}}
    \newcommand{\ErrorTok}[1]{\textcolor[rgb]{1.00,0.00,0.00}{\textbf{{#1}}}}
    \newcommand{\NormalTok}[1]{{#1}}
    
    % Additional commands for more recent versions of Pandoc
    \newcommand{\ConstantTok}[1]{\textcolor[rgb]{0.53,0.00,0.00}{{#1}}}
    \newcommand{\SpecialCharTok}[1]{\textcolor[rgb]{0.25,0.44,0.63}{{#1}}}
    \newcommand{\VerbatimStringTok}[1]{\textcolor[rgb]{0.25,0.44,0.63}{{#1}}}
    \newcommand{\SpecialStringTok}[1]{\textcolor[rgb]{0.73,0.40,0.53}{{#1}}}
    \newcommand{\ImportTok}[1]{{#1}}
    \newcommand{\DocumentationTok}[1]{\textcolor[rgb]{0.73,0.13,0.13}{\textit{{#1}}}}
    \newcommand{\AnnotationTok}[1]{\textcolor[rgb]{0.38,0.63,0.69}{\textbf{\textit{{#1}}}}}
    \newcommand{\CommentVarTok}[1]{\textcolor[rgb]{0.38,0.63,0.69}{\textbf{\textit{{#1}}}}}
    \newcommand{\VariableTok}[1]{\textcolor[rgb]{0.10,0.09,0.49}{{#1}}}
    \newcommand{\ControlFlowTok}[1]{\textcolor[rgb]{0.00,0.44,0.13}{\textbf{{#1}}}}
    \newcommand{\OperatorTok}[1]{\textcolor[rgb]{0.40,0.40,0.40}{{#1}}}
    \newcommand{\BuiltInTok}[1]{{#1}}
    \newcommand{\ExtensionTok}[1]{{#1}}
    \newcommand{\PreprocessorTok}[1]{\textcolor[rgb]{0.74,0.48,0.00}{{#1}}}
    \newcommand{\AttributeTok}[1]{\textcolor[rgb]{0.49,0.56,0.16}{{#1}}}
    \newcommand{\InformationTok}[1]{\textcolor[rgb]{0.38,0.63,0.69}{\textbf{\textit{{#1}}}}}
    \newcommand{\WarningTok}[1]{\textcolor[rgb]{0.38,0.63,0.69}{\textbf{\textit{{#1}}}}}
    
    
    % Define a nice break command that doesn't care if a line doesn't already
    % exist.
    \def\br{\hspace*{\fill} \\* }
    % Math Jax compatability definitions
	\def\TeX{\mbox{T\kern-.14em\lower.5ex\hbox{E}\kern-.115em X}}
	\def\LaTeX{\mbox{L\kern-.325em\raise.21em\hbox{$\scriptstyle{A}$}\kern-.17em}\TeX}

    \def\gt{>}
    \def\lt{<}
    % Document parameters
    \title{Numpy}
    
    
    

    % Pygments definitions
    
\makeatletter
\def\PY@reset{\let\PY@it=\relax \let\PY@bf=\relax%
    \let\PY@ul=\relax \let\PY@tc=\relax%
    \let\PY@bc=\relax \let\PY@ff=\relax}
\def\PY@tok#1{\csname PY@tok@#1\endcsname}
\def\PY@toks#1+{\ifx\relax#1\empty\else%
    \PY@tok{#1}\expandafter\PY@toks\fi}
\def\PY@do#1{\PY@bc{\PY@tc{\PY@ul{%
    \PY@it{\PY@bf{\PY@ff{#1}}}}}}}
\def\PY#1#2{\PY@reset\PY@toks#1+\relax+\PY@do{#2}}

\expandafter\def\csname PY@tok@gd\endcsname{\def\PY@tc##1{\textcolor[rgb]{0.63,0.00,0.00}{##1}}}
\expandafter\def\csname PY@tok@gu\endcsname{\let\PY@bf=\textbf\def\PY@tc##1{\textcolor[rgb]{0.50,0.00,0.50}{##1}}}
\expandafter\def\csname PY@tok@gt\endcsname{\def\PY@tc##1{\textcolor[rgb]{0.00,0.27,0.87}{##1}}}
\expandafter\def\csname PY@tok@gs\endcsname{\let\PY@bf=\textbf}
\expandafter\def\csname PY@tok@gr\endcsname{\def\PY@tc##1{\textcolor[rgb]{1.00,0.00,0.00}{##1}}}
\expandafter\def\csname PY@tok@cm\endcsname{\let\PY@it=\textit\def\PY@tc##1{\textcolor[rgb]{0.25,0.50,0.50}{##1}}}
\expandafter\def\csname PY@tok@vg\endcsname{\def\PY@tc##1{\textcolor[rgb]{0.10,0.09,0.49}{##1}}}
\expandafter\def\csname PY@tok@vi\endcsname{\def\PY@tc##1{\textcolor[rgb]{0.10,0.09,0.49}{##1}}}
\expandafter\def\csname PY@tok@vm\endcsname{\def\PY@tc##1{\textcolor[rgb]{0.10,0.09,0.49}{##1}}}
\expandafter\def\csname PY@tok@mh\endcsname{\def\PY@tc##1{\textcolor[rgb]{0.40,0.40,0.40}{##1}}}
\expandafter\def\csname PY@tok@cs\endcsname{\let\PY@it=\textit\def\PY@tc##1{\textcolor[rgb]{0.25,0.50,0.50}{##1}}}
\expandafter\def\csname PY@tok@ge\endcsname{\let\PY@it=\textit}
\expandafter\def\csname PY@tok@vc\endcsname{\def\PY@tc##1{\textcolor[rgb]{0.10,0.09,0.49}{##1}}}
\expandafter\def\csname PY@tok@il\endcsname{\def\PY@tc##1{\textcolor[rgb]{0.40,0.40,0.40}{##1}}}
\expandafter\def\csname PY@tok@go\endcsname{\def\PY@tc##1{\textcolor[rgb]{0.53,0.53,0.53}{##1}}}
\expandafter\def\csname PY@tok@cp\endcsname{\def\PY@tc##1{\textcolor[rgb]{0.74,0.48,0.00}{##1}}}
\expandafter\def\csname PY@tok@gi\endcsname{\def\PY@tc##1{\textcolor[rgb]{0.00,0.63,0.00}{##1}}}
\expandafter\def\csname PY@tok@gh\endcsname{\let\PY@bf=\textbf\def\PY@tc##1{\textcolor[rgb]{0.00,0.00,0.50}{##1}}}
\expandafter\def\csname PY@tok@ni\endcsname{\let\PY@bf=\textbf\def\PY@tc##1{\textcolor[rgb]{0.60,0.60,0.60}{##1}}}
\expandafter\def\csname PY@tok@nl\endcsname{\def\PY@tc##1{\textcolor[rgb]{0.63,0.63,0.00}{##1}}}
\expandafter\def\csname PY@tok@nn\endcsname{\let\PY@bf=\textbf\def\PY@tc##1{\textcolor[rgb]{0.00,0.00,1.00}{##1}}}
\expandafter\def\csname PY@tok@no\endcsname{\def\PY@tc##1{\textcolor[rgb]{0.53,0.00,0.00}{##1}}}
\expandafter\def\csname PY@tok@na\endcsname{\def\PY@tc##1{\textcolor[rgb]{0.49,0.56,0.16}{##1}}}
\expandafter\def\csname PY@tok@nb\endcsname{\def\PY@tc##1{\textcolor[rgb]{0.00,0.50,0.00}{##1}}}
\expandafter\def\csname PY@tok@nc\endcsname{\let\PY@bf=\textbf\def\PY@tc##1{\textcolor[rgb]{0.00,0.00,1.00}{##1}}}
\expandafter\def\csname PY@tok@nd\endcsname{\def\PY@tc##1{\textcolor[rgb]{0.67,0.13,1.00}{##1}}}
\expandafter\def\csname PY@tok@ne\endcsname{\let\PY@bf=\textbf\def\PY@tc##1{\textcolor[rgb]{0.82,0.25,0.23}{##1}}}
\expandafter\def\csname PY@tok@nf\endcsname{\def\PY@tc##1{\textcolor[rgb]{0.00,0.00,1.00}{##1}}}
\expandafter\def\csname PY@tok@si\endcsname{\let\PY@bf=\textbf\def\PY@tc##1{\textcolor[rgb]{0.73,0.40,0.53}{##1}}}
\expandafter\def\csname PY@tok@s2\endcsname{\def\PY@tc##1{\textcolor[rgb]{0.73,0.13,0.13}{##1}}}
\expandafter\def\csname PY@tok@nt\endcsname{\let\PY@bf=\textbf\def\PY@tc##1{\textcolor[rgb]{0.00,0.50,0.00}{##1}}}
\expandafter\def\csname PY@tok@nv\endcsname{\def\PY@tc##1{\textcolor[rgb]{0.10,0.09,0.49}{##1}}}
\expandafter\def\csname PY@tok@s1\endcsname{\def\PY@tc##1{\textcolor[rgb]{0.73,0.13,0.13}{##1}}}
\expandafter\def\csname PY@tok@dl\endcsname{\def\PY@tc##1{\textcolor[rgb]{0.73,0.13,0.13}{##1}}}
\expandafter\def\csname PY@tok@ch\endcsname{\let\PY@it=\textit\def\PY@tc##1{\textcolor[rgb]{0.25,0.50,0.50}{##1}}}
\expandafter\def\csname PY@tok@m\endcsname{\def\PY@tc##1{\textcolor[rgb]{0.40,0.40,0.40}{##1}}}
\expandafter\def\csname PY@tok@gp\endcsname{\let\PY@bf=\textbf\def\PY@tc##1{\textcolor[rgb]{0.00,0.00,0.50}{##1}}}
\expandafter\def\csname PY@tok@sh\endcsname{\def\PY@tc##1{\textcolor[rgb]{0.73,0.13,0.13}{##1}}}
\expandafter\def\csname PY@tok@ow\endcsname{\let\PY@bf=\textbf\def\PY@tc##1{\textcolor[rgb]{0.67,0.13,1.00}{##1}}}
\expandafter\def\csname PY@tok@sx\endcsname{\def\PY@tc##1{\textcolor[rgb]{0.00,0.50,0.00}{##1}}}
\expandafter\def\csname PY@tok@bp\endcsname{\def\PY@tc##1{\textcolor[rgb]{0.00,0.50,0.00}{##1}}}
\expandafter\def\csname PY@tok@c1\endcsname{\let\PY@it=\textit\def\PY@tc##1{\textcolor[rgb]{0.25,0.50,0.50}{##1}}}
\expandafter\def\csname PY@tok@fm\endcsname{\def\PY@tc##1{\textcolor[rgb]{0.00,0.00,1.00}{##1}}}
\expandafter\def\csname PY@tok@o\endcsname{\def\PY@tc##1{\textcolor[rgb]{0.40,0.40,0.40}{##1}}}
\expandafter\def\csname PY@tok@kc\endcsname{\let\PY@bf=\textbf\def\PY@tc##1{\textcolor[rgb]{0.00,0.50,0.00}{##1}}}
\expandafter\def\csname PY@tok@c\endcsname{\let\PY@it=\textit\def\PY@tc##1{\textcolor[rgb]{0.25,0.50,0.50}{##1}}}
\expandafter\def\csname PY@tok@mf\endcsname{\def\PY@tc##1{\textcolor[rgb]{0.40,0.40,0.40}{##1}}}
\expandafter\def\csname PY@tok@err\endcsname{\def\PY@bc##1{\setlength{\fboxsep}{0pt}\fcolorbox[rgb]{1.00,0.00,0.00}{1,1,1}{\strut ##1}}}
\expandafter\def\csname PY@tok@mb\endcsname{\def\PY@tc##1{\textcolor[rgb]{0.40,0.40,0.40}{##1}}}
\expandafter\def\csname PY@tok@ss\endcsname{\def\PY@tc##1{\textcolor[rgb]{0.10,0.09,0.49}{##1}}}
\expandafter\def\csname PY@tok@sr\endcsname{\def\PY@tc##1{\textcolor[rgb]{0.73,0.40,0.53}{##1}}}
\expandafter\def\csname PY@tok@mo\endcsname{\def\PY@tc##1{\textcolor[rgb]{0.40,0.40,0.40}{##1}}}
\expandafter\def\csname PY@tok@kd\endcsname{\let\PY@bf=\textbf\def\PY@tc##1{\textcolor[rgb]{0.00,0.50,0.00}{##1}}}
\expandafter\def\csname PY@tok@mi\endcsname{\def\PY@tc##1{\textcolor[rgb]{0.40,0.40,0.40}{##1}}}
\expandafter\def\csname PY@tok@kn\endcsname{\let\PY@bf=\textbf\def\PY@tc##1{\textcolor[rgb]{0.00,0.50,0.00}{##1}}}
\expandafter\def\csname PY@tok@cpf\endcsname{\let\PY@it=\textit\def\PY@tc##1{\textcolor[rgb]{0.25,0.50,0.50}{##1}}}
\expandafter\def\csname PY@tok@kr\endcsname{\let\PY@bf=\textbf\def\PY@tc##1{\textcolor[rgb]{0.00,0.50,0.00}{##1}}}
\expandafter\def\csname PY@tok@s\endcsname{\def\PY@tc##1{\textcolor[rgb]{0.73,0.13,0.13}{##1}}}
\expandafter\def\csname PY@tok@kp\endcsname{\def\PY@tc##1{\textcolor[rgb]{0.00,0.50,0.00}{##1}}}
\expandafter\def\csname PY@tok@w\endcsname{\def\PY@tc##1{\textcolor[rgb]{0.73,0.73,0.73}{##1}}}
\expandafter\def\csname PY@tok@kt\endcsname{\def\PY@tc##1{\textcolor[rgb]{0.69,0.00,0.25}{##1}}}
\expandafter\def\csname PY@tok@sc\endcsname{\def\PY@tc##1{\textcolor[rgb]{0.73,0.13,0.13}{##1}}}
\expandafter\def\csname PY@tok@sb\endcsname{\def\PY@tc##1{\textcolor[rgb]{0.73,0.13,0.13}{##1}}}
\expandafter\def\csname PY@tok@sa\endcsname{\def\PY@tc##1{\textcolor[rgb]{0.73,0.13,0.13}{##1}}}
\expandafter\def\csname PY@tok@k\endcsname{\let\PY@bf=\textbf\def\PY@tc##1{\textcolor[rgb]{0.00,0.50,0.00}{##1}}}
\expandafter\def\csname PY@tok@se\endcsname{\let\PY@bf=\textbf\def\PY@tc##1{\textcolor[rgb]{0.73,0.40,0.13}{##1}}}
\expandafter\def\csname PY@tok@sd\endcsname{\let\PY@it=\textit\def\PY@tc##1{\textcolor[rgb]{0.73,0.13,0.13}{##1}}}

\def\PYZbs{\char`\\}
\def\PYZus{\char`\_}
\def\PYZob{\char`\{}
\def\PYZcb{\char`\}}
\def\PYZca{\char`\^}
\def\PYZam{\char`\&}
\def\PYZlt{\char`\<}
\def\PYZgt{\char`\>}
\def\PYZsh{\char`\#}
\def\PYZpc{\char`\%}
\def\PYZdl{\char`\$}
\def\PYZhy{\char`\-}
\def\PYZsq{\char`\'}
\def\PYZdq{\char`\"}
\def\PYZti{\char`\~}
% for compatibility with earlier versions
\def\PYZat{@}
\def\PYZlb{[}
\def\PYZrb{]}
\makeatother


    % Exact colors from NB
    \definecolor{incolor}{rgb}{0.0, 0.0, 0.5}
    \definecolor{outcolor}{rgb}{0.545, 0.0, 0.0}



    
    % Prevent overflowing lines due to hard-to-break entities
    \sloppy 
    % Setup hyperref package
    \hypersetup{
      breaklinks=true,  % so long urls are correctly broken across lines
      colorlinks=true,
      urlcolor=urlcolor,
      linkcolor=linkcolor,
      citecolor=citecolor,
      }
    % Slightly bigger margins than the latex defaults
    
    \geometry{verbose,tmargin=1in,bmargin=1in,lmargin=1in,rmargin=1in}
    
    

    \begin{document}
    
    
    \maketitle
    
    

    
    \begin{Verbatim}[commandchars=\\\{\}]
{\color{incolor}In [{\color{incolor} }]:} \PY{c+c1}{\PYZsh{}Numpy stands for Numerical Python}
        \PY{c+c1}{\PYZsh{}It is an open source module of Python which provides fast mathematical computation on arrays and matrices. }
        \PY{k+kn}{import} \PY{n+nn}{numpy} \PY{k}{as} \PY{n+nn}{np}
        \PY{c+c1}{\PYZsh{}np is now the imported numpy package}
\end{Verbatim}


    \begin{Verbatim}[commandchars=\\\{\}]
{\color{incolor}In [{\color{incolor} }]:} \PY{c+c1}{\PYZsh{}differences between NumPy arrays and the standard Python sequences:}
        \PY{c+c1}{\PYZsh{} NumPy arrays have a fixed size at creation, unlike Python lists (which can grow dynamically).}
        \PY{c+c1}{\PYZsh{} Changing the size of an ndarray will create a new array and delete the original.}
        \PY{c+c1}{\PYZsh{}The elements in a NumPy array are all required to be of the same data type, and thus will be the same size in memory.}
        
        \PY{c+c1}{\PYZsh{} NumPy’s main object is the homogeneous multidimensional array. It is a table with same type elements, i.e, integers or string or characters (homogeneous), usually integers. }
        \PY{c+c1}{\PYZsh{} In a NumPy array, the number of dimensions is called the rank, and each dimension is called an axis. }
        \PY{c+c1}{\PYZsh{} So the rows are the first axis, and the columns are the second axis with a rank 2}
        \PY{c+c1}{\PYZsh{} For example, the coordinates of a point in 3D space [1, 2, 1] has one axis. }
        \PY{c+c1}{\PYZsh{} That axis has 3 elements in it, so we say it has a length of 3}
\end{Verbatim}


    \begin{Verbatim}[commandchars=\\\{\}]
{\color{incolor}In [{\color{incolor}36}]:} \PY{c+c1}{\PYZsh{} NumPy’s array class is called ndarray.}
         \PY{c+c1}{\PYZsh{} Attributes of ndarray}
         \PY{k+kn}{import} \PY{n+nn}{numpy} \PY{k}{as} \PY{n+nn}{np}
         \PY{c+c1}{\PYZsh{} To create sequences of numbers arrange(start,stop,step) is used}
         \PY{n}{a}\PY{o}{=}\PY{n}{np}\PY{o}{.}\PY{n}{arange}\PY{p}{(}\PY{l+m+mi}{15}\PY{p}{)}\PY{o}{.}\PY{n}{reshape}\PY{p}{(}\PY{l+m+mi}{3}\PY{p}{,}\PY{l+m+mi}{5}\PY{p}{)}
         \PY{n+nb}{print}\PY{p}{(}\PY{n}{a}\PY{p}{)}
         \PY{n+nb}{print}\PY{p}{(}\PY{n+nb}{type}\PY{p}{(}\PY{n}{a}\PY{p}{)}\PY{p}{)}
         \PY{c+c1}{\PYZsh{}the dimensions of the array. This is a tuple of integers indicating the size of the array in each dimension. For a matrix with n rows and m columns, shape will be (n,m)}
         \PY{n+nb}{print}\PY{p}{(}\PY{n}{a}\PY{o}{.}\PY{n}{shape}\PY{p}{)}
         \PY{c+c1}{\PYZsh{}the number of axes (dimensions) of the array.}
         \PY{n+nb}{print}\PY{p}{(}\PY{n}{a}\PY{o}{.}\PY{n}{ndim}\PY{p}{)}
         \PY{c+c1}{\PYZsh{}the total number of elements of the array. This is equal to the product of the elements of shape}
         \PY{n+nb}{print}\PY{p}{(}\PY{n}{a}\PY{o}{.}\PY{n}{size}\PY{p}{)}
         \PY{c+c1}{\PYZsh{}an object describing the type of the elements in the array. }
         \PY{n+nb}{print}\PY{p}{(}\PY{n}{a}\PY{o}{.}\PY{n}{dtype}\PY{p}{)}
         \PY{c+c1}{\PYZsh{}the size in bytes of each element of the array. }
         \PY{n+nb}{print}\PY{p}{(}\PY{n}{a}\PY{o}{.}\PY{n}{itemsize}\PY{p}{)}
         \PY{c+c1}{\PYZsh{}the buffer containing the actual elements of the array. }
         \PY{n+nb}{print}\PY{p}{(}\PY{n}{a}\PY{o}{.}\PY{n}{data}\PY{p}{)}
\end{Verbatim}


    \begin{Verbatim}[commandchars=\\\{\}]
[[ 0  1  2  3  4]
 [ 5  6  7  8  9]
 [10 11 12 13 14]]
<class 'numpy.ndarray'>
(3, 5)
2
15
int64
8
<memory at 0x7f1929a55dc8>

    \end{Verbatim}

    \begin{Verbatim}[commandchars=\\\{\}]
{\color{incolor}In [{\color{incolor}18}]:} \PY{c+c1}{\PYZsh{}Array Creation}
         \PY{k+kn}{import} \PY{n+nn}{numpy} \PY{k}{as} \PY{n+nn}{np}
         \PY{c+c1}{\PYZsh{}creating a vector}
         \PY{n}{a}\PY{o}{=}\PY{n}{np}\PY{o}{.}\PY{n}{array}\PY{p}{(}\PY{p}{[}\PY{l+m+mi}{2}\PY{p}{,}\PY{l+m+mi}{3}\PY{p}{,}\PY{l+m+mi}{4}\PY{p}{]}\PY{p}{)}
         \PY{n+nb}{print}\PY{p}{(}\PY{n}{a}\PY{p}{)}
         
         \PY{c+c1}{\PYZsh{}creating a matrix 2*3}
         \PY{n}{b}\PY{o}{=}\PY{n}{np}\PY{o}{.}\PY{n}{array}\PY{p}{(}\PY{p}{[}\PY{p}{(}\PY{l+m+mi}{1}\PY{p}{,}\PY{l+m+mi}{2}\PY{p}{,}\PY{l+m+mi}{3}\PY{p}{)}\PY{p}{,}\PY{p}{(}\PY{l+m+mi}{4}\PY{p}{,}\PY{l+m+mi}{5}\PY{p}{,}\PY{l+m+mi}{6}\PY{p}{)}\PY{p}{]}\PY{p}{)}
         \PY{n}{b}\PY{o}{=}\PY{n}{np}\PY{o}{.}\PY{n}{array}\PY{p}{(}\PY{p}{[}\PY{p}{[}\PY{l+m+mi}{1}\PY{p}{,}\PY{l+m+mi}{2}\PY{p}{,}\PY{l+m+mi}{3}\PY{p}{]}\PY{p}{,}\PY{p}{[}\PY{l+m+mi}{4}\PY{p}{,}\PY{l+m+mi}{5}\PY{p}{,}\PY{l+m+mi}{6}\PY{p}{]}\PY{p}{]}\PY{p}{)}
         \PY{n+nb}{print}\PY{p}{(}\PY{n}{b}\PY{p}{)}
         
         \PY{c+c1}{\PYZsh{} The type of the array can also be explicitly specified at creation time:}
         \PY{n}{c} \PY{o}{=} \PY{n}{np}\PY{o}{.}\PY{n}{array}\PY{p}{(} \PY{p}{[} \PY{p}{[}\PY{l+m+mi}{1}\PY{p}{,}\PY{l+m+mi}{2}\PY{p}{]}\PY{p}{,} \PY{p}{[}\PY{l+m+mi}{3}\PY{p}{,}\PY{l+m+mi}{4}\PY{p}{]} \PY{p}{]}\PY{p}{,} \PY{n}{dtype}\PY{o}{=}\PY{n+nb}{complex} \PY{p}{)}
         \PY{n+nb}{print}\PY{p}{(}\PY{n}{c}\PY{p}{)}
\end{Verbatim}


    \begin{Verbatim}[commandchars=\\\{\}]
[2 3 4]
[[1 2 3]
 [4 5 6]]
[[1.+0.j 2.+0.j]
 [3.+0.j 4.+0.j]]

    \end{Verbatim}

    \begin{Verbatim}[commandchars=\\\{\}]
{\color{incolor}In [{\color{incolor}39}]:} \PY{c+c1}{\PYZsh{} functions to create arrays with initial placeholder content.}
         \PY{c+c1}{\PYZsh{}filled with inital zeroes}
         \PY{n}{a}\PY{o}{=}\PY{n}{np}\PY{o}{.}\PY{n}{zeros}\PY{p}{(}\PY{p}{(}\PY{l+m+mi}{3}\PY{p}{,}\PY{l+m+mi}{4}\PY{p}{)}\PY{p}{)}
         \PY{n+nb}{print}\PY{p}{(}\PY{n}{a}\PY{p}{)}
         \PY{c+c1}{\PYZsh{}filled with inital ones}
         \PY{n}{b}\PY{o}{=}\PY{n}{np}\PY{o}{.}\PY{n}{ones}\PY{p}{(}\PY{p}{(}\PY{l+m+mi}{4}\PY{p}{,}\PY{l+m+mi}{3}\PY{p}{)}\PY{p}{,}\PY{n}{dtype}\PY{o}{=}\PY{n}{np}\PY{o}{.}\PY{n}{int16}\PY{p}{)}
         \PY{n+nb}{print}\PY{p}{(}\PY{n}{b}\PY{p}{)}
         \PY{c+c1}{\PYZsh{}filles with uninitailized random values}
         \PY{n}{c}\PY{o}{=}\PY{n}{np}\PY{o}{.}\PY{n}{empty}\PY{p}{(}\PY{p}{(}\PY{l+m+mi}{2}\PY{p}{,}\PY{l+m+mi}{3}\PY{p}{)}\PY{p}{)}
         \PY{n+nb}{print}\PY{p}{(}\PY{n}{c}\PY{p}{)}
         \PY{c+c1}{\PYZsh{}filles with initailized specified value}
         \PY{n}{d}\PY{o}{=}\PY{n}{np}\PY{o}{.}\PY{n}{full}\PY{p}{(}\PY{p}{(}\PY{l+m+mi}{3}\PY{p}{,}\PY{l+m+mi}{4}\PY{p}{)}\PY{p}{,}\PY{l+m+mi}{11}\PY{p}{)}
         \PY{n+nb}{print}\PY{p}{(}\PY{n}{d}\PY{p}{)}
         \PY{c+c1}{\PYZsh{}identity matrix creation}
         \PY{n}{e}\PY{o}{=}\PY{n}{np}\PY{o}{.}\PY{n}{eye}\PY{p}{(}\PY{l+m+mi}{3}\PY{p}{)}
         \PY{n+nb}{print}\PY{p}{(}\PY{n}{e}\PY{p}{)}
         \PY{c+c1}{\PYZsh{}diagonal matrix}
         \PY{n}{f}\PY{o}{=}\PY{n}{np}\PY{o}{.}\PY{n}{diag}\PY{p}{(}\PY{n}{np}\PY{o}{.}\PY{n}{array}\PY{p}{(}\PY{p}{[}\PY{l+m+mi}{1}\PY{p}{,}\PY{l+m+mi}{2}\PY{p}{,}\PY{l+m+mi}{3}\PY{p}{]}\PY{p}{)}\PY{p}{)}
         \PY{n+nb}{print}\PY{p}{(}\PY{n}{f}\PY{p}{)}
\end{Verbatim}


    \begin{Verbatim}[commandchars=\\\{\}]
[[0. 0. 0. 0.]
 [0. 0. 0. 0.]
 [0. 0. 0. 0.]]
[[1 1 1]
 [1 1 1]
 [1 1 1]
 [1 1 1]]
[[0.00708532 0.0348209  0.01863363]
 [0.69143149 0.46422209 0.67082615]]
[[11 11 11 11]
 [11 11 11 11]
 [11 11 11 11]]
[[1. 0. 0.]
 [0. 1. 0.]
 [0. 0. 1.]]
[[1 0 0]
 [0 2 0]
 [0 0 3]]

    \end{Verbatim}

    \begin{Verbatim}[commandchars=\\\{\}]
{\color{incolor}In [{\color{incolor}30}]:} \PY{c+c1}{\PYZsh{}When arange is used with floating point arguments, it is generally not possible to predict the number of elements obtained, }
         \PY{c+c1}{\PYZsh{} due to the finite floating point precision. For this reason, it is usually better to use the function linspace }
         \PY{c+c1}{\PYZsh{} that receives as an argument the number of elements that we want, instead of the step:}
         \PY{c+c1}{\PYZsh{} np.linspace(0,2,9) (start,end,no of points)}
         \PY{k+kn}{from} \PY{n+nn}{numpy} \PY{k}{import} \PY{n}{pi}
         \PY{n}{x}\PY{o}{=}\PY{n}{np}\PY{o}{.}\PY{n}{linspace}\PY{p}{(}\PY{l+m+mi}{0}\PY{p}{,}\PY{l+m+mi}{2}\PY{o}{*}\PY{n}{pi}\PY{p}{,}\PY{l+m+mi}{10}\PY{p}{)}
         \PY{n+nb}{print}\PY{p}{(}\PY{n}{x}\PY{p}{)}
\end{Verbatim}


    \begin{Verbatim}[commandchars=\\\{\}]
[0.         0.6981317  1.3962634  2.0943951  2.7925268  3.4906585
 4.1887902  4.88692191 5.58505361 6.28318531]

    \end{Verbatim}

    \begin{Verbatim}[commandchars=\\\{\}]
{\color{incolor}In [{\color{incolor}35}]:} \PY{c+c1}{\PYZsh{}generating random values}
         \PY{c+c1}{\PYZsh{}Create an array of the given shape and populate it with random samples from a uniform distribution over [0, 1).}
         \PY{n}{np}\PY{o}{.}\PY{n}{random}\PY{o}{.}\PY{n}{rand}\PY{p}{(}\PY{l+m+mi}{3}\PY{p}{,}\PY{l+m+mi}{2}\PY{p}{)}
         \PY{c+c1}{\PYZsh{}Return a sample (or samples) from the “standard normal” distribution.}
         \PY{c+c1}{\PYZsh{}For random samples from N(\PYZbs{}mu, \PYZbs{}sigma\PYZca{}2), use:}
         \PY{c+c1}{\PYZsh{} sigma * np.random.randn(...) + mu}
         \PY{n}{np}\PY{o}{.}\PY{n}{random}\PY{o}{.}\PY{n}{randn}\PY{p}{(}\PY{p}{)}
         \PY{c+c1}{\PYZsh{}Two\PYZhy{}by\PYZhy{}four array of samples from N(3, 6.25):}
         \PY{l+m+mf}{2.5} \PY{o}{*} \PY{n}{np}\PY{o}{.}\PY{n}{random}\PY{o}{.}\PY{n}{randn}\PY{p}{(}\PY{l+m+mi}{2}\PY{p}{,} \PY{l+m+mi}{4}\PY{p}{)} \PY{o}{+} \PY{l+m+mi}{3}
\end{Verbatim}


\begin{Verbatim}[commandchars=\\\{\}]
{\color{outcolor}Out[{\color{outcolor}35}]:} array([[0.08342544, 0.16351368, 1.78585464, 3.20856352],
                [3.17965976, 0.53570411, 3.03638892, 4.26555626]])
\end{Verbatim}
            
    \begin{Verbatim}[commandchars=\\\{\}]
{\color{incolor}In [{\color{incolor}41}]:} \PY{c+c1}{\PYZsh{}for reversing a sequence: }
         \PY{n}{a}\PY{o}{=}\PY{n}{np}\PY{o}{.}\PY{n}{arange}\PY{p}{(}\PY{l+m+mi}{1}\PY{p}{,}\PY{l+m+mi}{10}\PY{p}{,}\PY{l+m+mi}{2}\PY{p}{)}
         \PY{n+nb}{print}\PY{p}{(}\PY{n}{a}\PY{p}{)}
         \PY{n+nb}{print}\PY{p}{(}\PY{n}{a}\PY{p}{[}\PY{p}{:}\PY{p}{:}\PY{o}{\PYZhy{}}\PY{l+m+mi}{1}\PY{p}{]}\PY{p}{)}
         \PY{c+c1}{\PYZsh{}A slicing operation creates a view on the original array, which is just a way of accessing array data. Thus the originalarrayisnotcopiedinmemory. }
         \PY{c+c1}{\PYZsh{}You can usenp.may\PYZus{}share\PYZus{}memory()to check if two arrays share the same memory block}
\end{Verbatim}


    \begin{Verbatim}[commandchars=\\\{\}]
[1 3 5 7 9]
[9 7 5 3 1]

    \end{Verbatim}

    \begin{Verbatim}[commandchars=\\\{\}]
{\color{incolor}In [{\color{incolor}46}]:} \PY{c+c1}{\PYZsh{}computing prime numbers in a give range}
         \PY{n}{start}\PY{o}{=}\PY{n+nb}{int}\PY{p}{(}\PY{n+nb}{input}\PY{p}{(}\PY{l+s+s2}{\PYZdq{}}\PY{l+s+s2}{Enter start number:}\PY{l+s+s2}{\PYZdq{}}\PY{p}{)}\PY{p}{)}
         \PY{n}{end}\PY{o}{=}\PY{n+nb}{int}\PY{p}{(}\PY{n+nb}{input}\PY{p}{(}\PY{l+s+s2}{\PYZdq{}}\PY{l+s+s2}{Enter end number:}\PY{l+s+s2}{\PYZdq{}}\PY{p}{)}\PY{p}{)}
         \PY{n}{length}\PY{o}{=}\PY{n}{end}\PY{o}{\PYZhy{}}\PY{n}{start}
         \PY{n}{is\PYZus{}prime}\PY{o}{=}\PY{n}{np}\PY{o}{.}\PY{n}{ones}\PY{p}{(}\PY{p}{(}\PY{n}{length}\PY{p}{,}\PY{p}{)}\PY{p}{,}\PY{n}{dtype}\PY{o}{=}\PY{n+nb}{bool}\PY{p}{)}
         \PY{n}{N\PYZus{}max}\PY{o}{=}\PY{n+nb}{int}\PY{p}{(}\PY{n}{np}\PY{o}{.}\PY{n}{sqrt}\PY{p}{(}\PY{n+nb}{len}\PY{p}{(}\PY{n}{is\PYZus{}prime}\PY{p}{)}\PY{o}{\PYZhy{}}\PY{l+m+mi}{1}\PY{p}{)}\PY{p}{)}
         \PY{k}{for} \PY{n}{j} \PY{o+ow}{in} \PY{n+nb}{range}\PY{p}{(}\PY{l+m+mi}{2}\PY{p}{,}\PY{n}{N\PYZus{}max}\PY{o}{+}\PY{l+m+mi}{1}\PY{p}{)}\PY{p}{:}
             \PY{n}{is\PYZus{}prime}\PY{p}{[}\PY{l+m+mi}{2}\PY{o}{*}\PY{n}{j}\PY{p}{:}\PY{p}{:}\PY{n}{j}\PY{p}{]}\PY{o}{=}\PY{k+kc}{False}
         \PY{n+nb}{print}\PY{p}{(}\PY{n}{is\PYZus{}prime}\PY{p}{)}
\end{Verbatim}


    \begin{Verbatim}[commandchars=\\\{\}]
Enter start number:0
Enter end number:10
[ True  True  True  True False  True False  True False False]

    \end{Verbatim}

    \begin{Verbatim}[commandchars=\\\{\}]
{\color{incolor}In [{\color{incolor}63}]:} \PY{c+c1}{\PYZsh{}Operations in Numpy}
         \PY{c+c1}{\PYZsh{}Elementwise Operations}
         \PY{c+c1}{\PYZsh{}with scalars}
         \PY{n}{a}\PY{o}{=}\PY{n}{np}\PY{o}{.}\PY{n}{array}\PY{p}{(}\PY{p}{[}\PY{l+m+mi}{1}\PY{p}{,}\PY{l+m+mi}{2}\PY{p}{,}\PY{l+m+mi}{3}\PY{p}{,}\PY{l+m+mi}{4}\PY{p}{]}\PY{p}{)}
         \PY{n+nb}{print}\PY{p}{(}\PY{n}{a}\PY{o}{+}\PY{l+m+mi}{1}\PY{p}{)}
         \PY{n+nb}{print}\PY{p}{(}\PY{l+m+mi}{2}\PY{o}{*}\PY{o}{*}\PY{n}{a}\PY{p}{)}
         \PY{c+c1}{\PYZsh{}numerical Operations}
         \PY{n}{b} \PY{o}{=} \PY{n}{np}\PY{o}{.}\PY{n}{arange}\PY{p}{(} \PY{l+m+mi}{4} \PY{p}{)}\PY{o}{+}\PY{l+m+mi}{1}
         \PY{n+nb}{print}\PY{p}{(}\PY{n}{b}\PY{p}{)}
         \PY{n}{a}\PY{o}{\PYZhy{}}\PY{n}{b}
         \PY{n}{a}\PY{o}{*}\PY{n}{b}\PY{c+c1}{\PYZsh{}not matrix multiplication}
         \PY{n}{a}\PY{o}{\PYZhy{}}\PY{n}{b}
         \PY{c+c1}{\PYZsh{}for matrix multiplication}
         \PY{n}{c}\PY{o}{=}\PY{n}{a}\PY{o}{.}\PY{n}{dot}\PY{p}{(}\PY{n}{b}\PY{p}{)} \PY{c+c1}{\PYZsh{}or np.dot(a,b)}
         \PY{n+nb}{print}\PY{p}{(}\PY{n}{c}\PY{p}{)}
         \PY{n}{a}\PY{o}{\PYZlt{}}\PY{l+m+mi}{3}
         \PY{c+c1}{\PYZsh{}Some operations, such as += and *=, act in place to modify an existing array rather than create a new one.}
         \PY{n}{b} \PY{o}{+}\PY{o}{=} \PY{n}{a}
         \PY{n+nb}{print}\PY{p}{(}\PY{n}{b}\PY{p}{)}
\end{Verbatim}


    \begin{Verbatim}[commandchars=\\\{\}]
[2 3 4 5]
[ 2  4  8 16]
[1 2 3 4]
30
[2 4 6 8]

    \end{Verbatim}

    \begin{Verbatim}[commandchars=\\\{\}]
{\color{incolor}In [{\color{incolor}70}]:} \PY{c+c1}{\PYZsh{}comparisions OPerations}
         \PY{n}{a}\PY{o}{==}\PY{n}{b}
         \PY{n}{a}\PY{o}{\PYZgt{}}\PY{n}{b}
         \PY{n}{np}\PY{o}{.}\PY{n}{array\PYZus{}equal}\PY{p}{(}\PY{n}{a}\PY{p}{,}\PY{n}{b}\PY{p}{)}
         \PY{c+c1}{\PYZsh{}logical Operations}
         \PY{n}{np}\PY{o}{.}\PY{n}{logical\PYZus{}or}\PY{p}{(}\PY{n}{a}\PY{p}{,}\PY{n}{b}\PY{p}{)}
         \PY{n}{np}\PY{o}{.}\PY{n}{logical\PYZus{}and}\PY{p}{(}\PY{n}{a}\PY{p}{,}\PY{n}{b}\PY{p}{)}
         \PY{c+c1}{\PYZsh{}otheroperations}
         \PY{n}{np}\PY{o}{.}\PY{n}{log}\PY{p}{(}\PY{l+m+mi}{15}\PY{p}{)}
         \PY{n}{np}\PY{o}{.}\PY{n}{exp}\PY{p}{(}\PY{n}{a}\PY{p}{)}
\end{Verbatim}


\begin{Verbatim}[commandchars=\\\{\}]
{\color{outcolor}Out[{\color{outcolor}70}]:} array([ 2.71828183,  7.3890561 , 20.08553692, 54.59815003])
\end{Verbatim}
            
    \begin{Verbatim}[commandchars=\\\{\}]
{\color{incolor}In [{\color{incolor}84}]:} \PY{c+c1}{\PYZsh{}BAsic Reductions}
         \PY{n}{x}\PY{o}{=}\PY{n}{np}\PY{o}{.}\PY{n}{array}\PY{p}{(}\PY{p}{[}\PY{p}{[}\PY{l+m+mi}{1}\PY{p}{,}\PY{l+m+mi}{1}\PY{p}{]}\PY{p}{,}\PY{p}{[}\PY{l+m+mi}{2}\PY{p}{,}\PY{l+m+mi}{2}\PY{p}{]}\PY{p}{]}\PY{p}{)}
         \PY{n}{np}\PY{o}{.}\PY{n}{sum}\PY{p}{(}\PY{n}{x}\PY{p}{)}
         \PY{n}{x}\PY{o}{.}\PY{n}{sum}\PY{p}{(}\PY{n}{axis}\PY{o}{=}\PY{l+m+mi}{0}\PY{p}{)}\PY{c+c1}{\PYZsh{}columns}
         \PY{n}{x}\PY{o}{.}\PY{n}{sum}\PY{p}{(}\PY{n}{axis}\PY{o}{=}\PY{l+m+mi}{1}\PY{p}{)}\PY{c+c1}{\PYZsh{}rows}
         \PY{n}{z}\PY{o}{=}\PY{n}{np}\PY{o}{.}\PY{n}{array}\PY{p}{(}\PY{p}{[}\PY{l+m+mi}{1}\PY{p}{,}\PY{l+m+mi}{2}\PY{p}{,}\PY{l+m+mi}{34}\PY{p}{,}\PY{l+m+mi}{5}\PY{p}{,}\PY{l+m+mi}{6}\PY{p}{]}\PY{p}{)}
         \PY{n+nb}{print}\PY{p}{(}\PY{n}{z}\PY{o}{.}\PY{n}{min}\PY{p}{(}\PY{p}{)}\PY{p}{)}
         \PY{n+nb}{print}\PY{p}{(}\PY{n}{z}\PY{o}{.}\PY{n}{max}\PY{p}{(}\PY{p}{)}\PY{p}{)}
         \PY{n+nb}{print}\PY{p}{(}\PY{n}{z}\PY{o}{.}\PY{n}{argmin}\PY{p}{(}\PY{p}{)}\PY{p}{)}\PY{c+c1}{\PYZsh{}index of the min element}
         \PY{n+nb}{print}\PY{p}{(}\PY{n}{z}\PY{o}{.}\PY{n}{argmax}\PY{p}{(}\PY{p}{)}\PY{p}{)}
         \PY{n+nb}{print}\PY{p}{(}\PY{n}{z}\PY{o}{.}\PY{n}{mean}\PY{p}{(}\PY{p}{)}\PY{p}{)}
         \PY{n+nb}{print}\PY{p}{(}\PY{n}{np}\PY{o}{.}\PY{n}{median}\PY{p}{(}\PY{n}{z}\PY{p}{)}\PY{p}{)}
         \PY{n+nb}{print}\PY{p}{(}\PY{n}{z}\PY{o}{.}\PY{n}{std}\PY{p}{(}\PY{p}{)}\PY{p}{)}
\end{Verbatim}


    \begin{Verbatim}[commandchars=\\\{\}]
1
34
0
2
9.6
5.0
12.338557452149743

    \end{Verbatim}

    \begin{Verbatim}[commandchars=\\\{\}]
{\color{incolor}In [{\color{incolor}91}]:} \PY{c+c1}{\PYZsh{}shape manipulation}
         \PY{c+c1}{\PYZsh{}the following three commands all return a modified array, but do not change the original array:}
         \PY{n}{a} \PY{o}{=} \PY{n}{np}\PY{o}{.}\PY{n}{floor}\PY{p}{(}\PY{l+m+mi}{10}\PY{o}{*}\PY{n}{np}\PY{o}{.}\PY{n}{random}\PY{o}{.}\PY{n}{random}\PY{p}{(}\PY{p}{(}\PY{l+m+mi}{3}\PY{p}{,}\PY{l+m+mi}{4}\PY{p}{)}\PY{p}{)}\PY{p}{)}
         \PY{n+nb}{print}\PY{p}{(}\PY{n}{a}\PY{o}{.}\PY{n}{ravel}\PY{p}{(}\PY{p}{)}\PY{p}{)}\PY{c+c1}{\PYZsh{}returns a vector}
         \PY{n+nb}{print}\PY{p}{(}\PY{n}{a}\PY{o}{.}\PY{n}{reshape}\PY{p}{(}\PY{l+m+mi}{6}\PY{p}{,}\PY{l+m+mi}{2}\PY{p}{)}\PY{p}{)} \PY{c+c1}{\PYZsh{} returns the array with a modified shape}
         \PY{n}{a}\PY{o}{.}\PY{n}{T}\PY{c+c1}{\PYZsh{} returns the array, transposed}
         \PY{n+nb}{print}\PY{p}{(}\PY{n}{a}\PY{p}{)}
         \PY{c+c1}{\PYZsh{}The reshape function returns its argument with a modified shape, whereas the ndarray.resize method modifies the array itself:}
         \PY{n}{a}\PY{o}{.}\PY{n}{resize}\PY{p}{(}\PY{p}{(}\PY{l+m+mi}{2}\PY{p}{,}\PY{l+m+mi}{6}\PY{p}{)}\PY{p}{)}
         \PY{n+nb}{print}\PY{p}{(}\PY{n}{a}\PY{p}{)}
\end{Verbatim}


    \begin{Verbatim}[commandchars=\\\{\}]
[1. 5. 2. 7. 7. 3. 6. 9. 1. 9. 5. 1.]
[[1. 5.]
 [2. 7.]
 [7. 3.]
 [6. 9.]
 [1. 9.]
 [5. 1.]]
[[1. 5. 2. 7.]
 [7. 3. 6. 9.]
 [1. 9. 5. 1.]]
[[1. 5. 2. 7. 7. 3.]
 [6. 9. 1. 9. 5. 1.]]

    \end{Verbatim}

    \begin{Verbatim}[commandchars=\\\{\}]
{\color{incolor}In [{\color{incolor}94}]:} \PY{c+c1}{\PYZsh{} Basic operations on numpy arrays(addition,etc.) are elementwise }
         \PY{c+c1}{\PYZsh{} Nevertheless,It’s also possible to do operations on arrays of different sizes if NumPy can transform these arrays so that they all have the samesize: this conversion is called broadcasting.}
         \PY{c+c1}{\PYZsh{}adding a dimension}
         \PY{n}{z}\PY{o}{=}\PY{n}{np}\PY{o}{.}\PY{n}{array}\PY{p}{(}\PY{p}{[}\PY{l+m+mi}{1}\PY{p}{,}\PY{l+m+mi}{2}\PY{p}{,}\PY{l+m+mi}{3}\PY{p}{,}\PY{l+m+mi}{4}\PY{p}{,}\PY{l+m+mi}{5}\PY{p}{]}\PY{p}{)}
         \PY{n+nb}{print}\PY{p}{(}\PY{n}{z}\PY{p}{)}
         \PY{n}{z}\PY{p}{[}\PY{p}{:}\PY{p}{,} \PY{n}{np}\PY{o}{.}\PY{n}{newaxis}\PY{p}{]}
         \PY{n+nb}{print}\PY{p}{(}\PY{n}{z}\PY{p}{)}
         \PY{n}{z}\PY{p}{[}\PY{n}{np}\PY{o}{.}\PY{n}{newaxis}\PY{p}{,}\PY{p}{:}\PY{p}{]}
         \PY{n+nb}{print}\PY{p}{(}\PY{n}{z}\PY{p}{)}
\end{Verbatim}


    \begin{Verbatim}[commandchars=\\\{\}]
[1 2 3 4 5]
[1 2 3 4 5]
[1 2 3 4 5]

    \end{Verbatim}

    \begin{Verbatim}[commandchars=\\\{\}]
{\color{incolor}In [{\color{incolor}96}]:} \PY{n}{a} \PY{o}{=} \PY{n}{np}\PY{o}{.}\PY{n}{array}\PY{p}{(}\PY{p}{[}\PY{p}{[}\PY{l+m+mi}{4}\PY{p}{,} \PY{l+m+mi}{3}\PY{p}{,} \PY{l+m+mi}{5}\PY{p}{]}\PY{p}{,} \PY{p}{[}\PY{l+m+mi}{1}\PY{p}{,} \PY{l+m+mi}{2}\PY{p}{,} \PY{l+m+mi}{1}\PY{p}{]}\PY{p}{]}\PY{p}{)} 
         \PY{n}{b} \PY{o}{=} \PY{n}{np}\PY{o}{.}\PY{n}{sort}\PY{p}{(}\PY{n}{a}\PY{p}{,} \PY{n}{axis}\PY{o}{=}\PY{l+m+mi}{1}\PY{p}{)} \PY{c+c1}{\PYZsh{}sorting each row separtely}
         \PY{n+nb}{print}\PY{p}{(}\PY{n}{b}\PY{p}{)}
         \PY{n}{a}\PY{o}{.}\PY{n}{sort}\PY{p}{(}\PY{n}{axis}\PY{o}{=}\PY{l+m+mi}{1}\PY{p}{)}\PY{c+c1}{\PYZsh{}in\PYZhy{}place sorting}
         \PY{n+nb}{print}\PY{p}{(}\PY{n}{a}\PY{p}{)}
\end{Verbatim}


    \begin{Verbatim}[commandchars=\\\{\}]
[[3 4 5]
 [1 1 2]]
[[3 4 5]
 [1 1 2]]

    \end{Verbatim}

    \begin{Verbatim}[commandchars=\\\{\}]
{\color{incolor}In [{\color{incolor}100}]:} \PY{c+c1}{\PYZsh{}structured data type}
          \PY{n}{samples} \PY{o}{=} \PY{n}{np}\PY{o}{.}\PY{n}{zeros}\PY{p}{(}\PY{p}{(}\PY{l+m+mi}{6}\PY{p}{,}\PY{p}{)}\PY{p}{,} \PY{n}{dtype}\PY{o}{=}\PY{p}{[}\PY{p}{(}\PY{l+s+s1}{\PYZsq{}}\PY{l+s+s1}{sensor\PYZus{}code}\PY{l+s+s1}{\PYZsq{}}\PY{p}{,} \PY{l+s+s1}{\PYZsq{}}\PY{l+s+s1}{S4}\PY{l+s+s1}{\PYZsq{}}\PY{p}{)}\PY{p}{,}\PY{p}{(}\PY{l+s+s1}{\PYZsq{}}\PY{l+s+s1}{position}\PY{l+s+s1}{\PYZsq{}}\PY{p}{,} \PY{n+nb}{float}\PY{p}{)}\PY{p}{,} \PY{p}{(}\PY{l+s+s1}{\PYZsq{}}\PY{l+s+s1}{value}\PY{l+s+s1}{\PYZsq{}}\PY{p}{,} \PY{n+nb}{float}\PY{p}{)}\PY{p}{]}\PY{p}{)}
          \PY{n+nb}{print}\PY{p}{(}\PY{n}{samples}\PY{p}{)}
          \PY{n}{samples}\PY{p}{[}\PY{p}{:}\PY{p}{]} \PY{o}{=} \PY{p}{[}\PY{p}{(}\PY{l+s+s1}{\PYZsq{}}\PY{l+s+s1}{ALFA}\PY{l+s+s1}{\PYZsq{}}\PY{p}{,} \PY{l+m+mi}{1}\PY{p}{,} \PY{l+m+mf}{0.37}\PY{p}{)}\PY{p}{,} \PY{p}{(}\PY{l+s+s1}{\PYZsq{}}\PY{l+s+s1}{BETA}\PY{l+s+s1}{\PYZsq{}}\PY{p}{,} \PY{l+m+mi}{1}\PY{p}{,} \PY{l+m+mf}{0.11}\PY{p}{)}\PY{p}{,} \PY{p}{(}\PY{l+s+s1}{\PYZsq{}}\PY{l+s+s1}{TAU}\PY{l+s+s1}{\PYZsq{}}\PY{p}{,} \PY{l+m+mi}{1}\PY{p}{,} \PY{l+m+mf}{0.13}\PY{p}{)}\PY{p}{,}\PY{p}{(}\PY{l+s+s1}{\PYZsq{}}\PY{l+s+s1}{ALFA}\PY{l+s+s1}{\PYZsq{}}\PY{p}{,} \PY{l+m+mf}{1.5}\PY{p}{,} \PY{l+m+mf}{0.37}\PY{p}{)}\PY{p}{,} \PY{p}{(}\PY{l+s+s1}{\PYZsq{}}\PY{l+s+s1}{ALFA}\PY{l+s+s1}{\PYZsq{}}\PY{p}{,} \PY{l+m+mi}{3}\PY{p}{,} \PY{l+m+mf}{0.11}\PY{p}{)}\PY{p}{,} \PY{p}{(}\PY{l+s+s1}{\PYZsq{}}\PY{l+s+s1}{TAU}\PY{l+s+s1}{\PYZsq{}}\PY{p}{,} \PY{l+m+mf}{1.2}\PY{p}{,} \PY{l+m+mf}{0.13}\PY{p}{)}\PY{p}{]}
          \PY{n+nb}{print}\PY{p}{(}\PY{n}{samples}\PY{p}{)}
          \PY{n+nb}{print}\PY{p}{(}\PY{n}{samples}\PY{p}{[}\PY{l+s+s1}{\PYZsq{}}\PY{l+s+s1}{sensor\PYZus{}code}\PY{l+s+s1}{\PYZsq{}}\PY{p}{]}\PY{p}{)}
          \PY{n}{samples}\PY{p}{[}\PY{l+m+mi}{0}\PY{p}{]}
\end{Verbatim}


    \begin{Verbatim}[commandchars=\\\{\}]
[(b'', 0., 0.) (b'', 0., 0.) (b'', 0., 0.) (b'', 0., 0.) (b'', 0., 0.)
 (b'', 0., 0.)]
[(b'ALFA', 1. , 0.37) (b'BETA', 1. , 0.11) (b'TAU', 1. , 0.13)
 (b'ALFA', 1.5, 0.37) (b'ALFA', 3. , 0.11) (b'TAU', 1.2, 0.13)]
[b'ALFA' b'BETA' b'TAU' b'ALFA' b'ALFA' b'TAU']

    \end{Verbatim}

\begin{Verbatim}[commandchars=\\\{\}]
{\color{outcolor}Out[{\color{outcolor}100}]:} (b'ALFA', 1., 0.37)
\end{Verbatim}
            
    \begin{Verbatim}[commandchars=\\\{\}]
{\color{incolor}In [{\color{incolor}107}]:} \PY{c+c1}{\PYZsh{}polynomials}
          \PY{c+c1}{\PYZsh{}3x**2+2*x\PYZhy{}1}
          \PY{n}{p}\PY{o}{=}\PY{n}{np}\PY{o}{.}\PY{n}{poly1d}\PY{p}{(}\PY{p}{[}\PY{l+m+mi}{3}\PY{p}{,}\PY{l+m+mi}{2}\PY{p}{,}\PY{o}{\PYZhy{}}\PY{l+m+mi}{1}\PY{p}{]}\PY{p}{)}
          \PY{n+nb}{print}\PY{p}{(}\PY{n}{p}\PY{p}{)}
          \PY{n+nb}{print}\PY{p}{(}\PY{n}{p}\PY{p}{(}\PY{l+m+mi}{0}\PY{p}{)}\PY{p}{)}
          \PY{n+nb}{print}\PY{p}{(}\PY{n}{p}\PY{o}{.}\PY{n}{roots}\PY{p}{)}
          \PY{n+nb}{print}\PY{p}{(}\PY{n}{p}\PY{o}{.}\PY{n}{order}\PY{p}{)}
          \PY{n}{p} \PY{o}{=} \PY{n}{np}\PY{o}{.}\PY{n}{polynomial}\PY{o}{.}\PY{n}{Polynomial}\PY{p}{(}\PY{p}{[}\PY{o}{\PYZhy{}}\PY{l+m+mi}{1}\PY{p}{,} \PY{l+m+mi}{2}\PY{p}{,} \PY{l+m+mi}{3}\PY{p}{]}\PY{p}{)} \PY{c+c1}{\PYZsh{} coefs in different order! }
          \PY{n+nb}{print}\PY{p}{(}\PY{n}{p}\PY{o}{.}\PY{n}{roots}\PY{p}{(}\PY{p}{)}\PY{p}{)}
          \PY{n+nb}{print}\PY{p}{(}\PY{n}{p}\PY{o}{.}\PY{n}{degree}\PY{p}{(}\PY{p}{)}\PY{p}{)}
\end{Verbatim}


    \begin{Verbatim}[commandchars=\\\{\}]
   2
3 x + 2 x - 1
-1
[-1.          0.33333333]
2
[-1.          0.33333333]
2

    \end{Verbatim}

    \begin{Verbatim}[commandchars=\\\{\}]
{\color{incolor}In [{\color{incolor} }]:} \PY{c+c1}{\PYZsh{}loading datafiles}
        \PY{n}{data} \PY{o}{=} \PY{n}{np}\PY{o}{.}\PY{n}{loadtxt}\PY{p}{(}\PY{l+s+s1}{\PYZsq{}}\PY{l+s+s1}{data/populations.txt}\PY{l+s+s1}{\PYZsq{}}\PY{p}{)} 
        \PY{n+nb}{print}\PY{p}{(}\PY{n}{data}\PY{p}{)}
\end{Verbatim}


    \begin{Verbatim}[commandchars=\\\{\}]
{\color{incolor}In [{\color{incolor}111}]:} \PY{n}{a} \PY{o}{=} \PY{n}{np}\PY{o}{.}\PY{n}{array}\PY{p}{(}\PY{p}{[}\PY{p}{[}\PY{l+m+mf}{1.0}\PY{p}{,} \PY{l+m+mf}{2.0}\PY{p}{]}\PY{p}{,} \PY{p}{[}\PY{l+m+mf}{3.0}\PY{p}{,} \PY{l+m+mf}{4.0}\PY{p}{]}\PY{p}{]}\PY{p}{)}
          \PY{n+nb}{print}\PY{p}{(}\PY{n}{a}\PY{p}{)}
          \PY{n+nb}{print}\PY{p}{(}\PY{n}{a}\PY{o}{.}\PY{n}{transpose}\PY{p}{(}\PY{p}{)}\PY{p}{)}
          \PY{n+nb}{print}\PY{p}{(}\PY{n}{np}\PY{o}{.}\PY{n}{linalg}\PY{o}{.}\PY{n}{inv}\PY{p}{(}\PY{n}{a}\PY{p}{)}\PY{p}{)}
          \PY{n}{u} \PY{o}{=} \PY{n}{np}\PY{o}{.}\PY{n}{eye}\PY{p}{(}\PY{l+m+mi}{2}\PY{p}{)}
          \PY{n+nb}{print}\PY{p}{(}\PY{n}{np}\PY{o}{.}\PY{n}{trace}\PY{p}{(}\PY{n}{u}\PY{p}{)} \PY{p}{)} \PY{c+c1}{\PYZsh{} trace}
          \PY{n}{y} \PY{o}{=} \PY{n}{np}\PY{o}{.}\PY{n}{array}\PY{p}{(}\PY{p}{[}\PY{p}{[}\PY{l+m+mf}{5.}\PY{p}{]}\PY{p}{,} \PY{p}{[}\PY{l+m+mf}{7.}\PY{p}{]}\PY{p}{]}\PY{p}{)}
          \PY{n+nb}{print}\PY{p}{(}\PY{n}{np}\PY{o}{.}\PY{n}{linalg}\PY{o}{.}\PY{n}{solve}\PY{p}{(}\PY{n}{a}\PY{p}{,} \PY{n}{y}\PY{p}{)}\PY{p}{)}
\end{Verbatim}


    \begin{Verbatim}[commandchars=\\\{\}]
[[1. 2.]
 [3. 4.]]
[[1. 3.]
 [2. 4.]]
[[-2.   1. ]
 [ 1.5 -0.5]]
2.0
[[-3.]
 [ 4.]]

    \end{Verbatim}

    \begin{Verbatim}[commandchars=\\\{\}]
{\color{incolor}In [{\color{incolor}112}]:} \PY{c+c1}{\PYZsh{}indices() will create a set of arrays (stacked as a one\PYZhy{}higher dimensioned array), one per dimension with each representing variation in that dimension}
          \PY{n}{np}\PY{o}{.}\PY{n}{indices}\PY{p}{(}\PY{p}{(}\PY{l+m+mi}{3}\PY{p}{,}\PY{l+m+mi}{3}\PY{p}{)}\PY{p}{)}
\end{Verbatim}


\begin{Verbatim}[commandchars=\\\{\}]
{\color{outcolor}Out[{\color{outcolor}112}]:} array([[[0, 0, 0],
                  [1, 1, 1],
                  [2, 2, 2]],
          
                 [[0, 1, 2],
                  [0, 1, 2],
                  [0, 1, 2]]])
\end{Verbatim}
            

    % Add a bibliography block to the postdoc
    
    
    
    \end{document}
